% TO-DO:

\documentclass[orivec]{llncs}
\usepackage{graphicx}
\usepackage{float}
\usepackage[most]{tcolorbox}% for wrapping example in color box
\usepackage{wrapfig}		% wrap figure beside text, used in example
% \usepackage{tikz-cd}		% commutative diagrams
% \usepackage{amsfonts}
\usepackage[normalem]{ulem}	% underline with line breaks: /uline
\usepackage{enumitem}       % for using (A),(B),(C) in items...
% \usepackage{amsmath}		% for "cases"
% \usepackage{amsfonts}		% for frakur fonts
% \usepackage{mathrsfs}		% for curly "E" error symbol
% \usepackage{amssymb}		% for \multimap, \updownarrow, \bigstar
% \usepackage{turnstile}		% longer turnstiles
\usepackage{sectsty}		% change section color
\usepackage{hyperref}		% refs, links become clickable
\usepackage{url}			% for urls in bibliography
\usepackage[normalem]{ulem} % underline unbroken with \uline
\usepackage[numbers,sectionbib]{natbib}% if we use \package{url} we need to use natbib style
% \usepackage{unicode-math}
\usepackage{titlesec}
\usepackage{xpatch}
\usepackage{changepage}

%\def\chinchin{yes}          % ********** 用中文 *********
% *************** Delete when not using Chinese or colors **********************
\ifdefined\chinchin
\usepackage{xeCJK}
\setCJKmainfont[BoldFont=SimHei,ItalicFont=KaiTi]{SimSun}
\newcommand{\cc}[2]{#1}
\else
\newcommand{\cc}[2]{#2}
\fi
\usepackage{color}
%\newcommand{\emp}[1]{\textbf{\textcolor{blue}{#1}}}
\newcommand{\emp}[1]{\textbf{#1}}

% ***************** Formatting of sections *********************
\setcounter{secnumdepth}{3}
\sectionfont{\bfseries\color{blue}} 
% \subsectionfont{\color{blue}} 
\subsubsectionfont{\color{blue}} 
\definecolor{green}{rgb}{0,0.7,0}
\definecolor{grey}{rgb}{0.95,0.95,0.95}

% ********************* indentation of subsubsections **********************
\let\subsubaw\adjustwidth
\xpatchcmd{\subsubaw}
  {\topsep}
  {\labelwidth=0pt \labelsep=0pt \topsep}
  {}{}
\xpatchcmd{\subsubaw}
  {\item[]}
  {\refstepcounter{subsubsection}\item[\bfseries\color{blue}\thesubsubsection.\hspace{3pt} ]}
  {}{}
\newenvironment{subsub}
  {\par\addvspace{3.25ex plus 1ex minus .2ex}%
    \subsubaw{\parindent}{0pt}}
  {\endadjustwidth}

\titleformat{\subsection}[runin]{\large\bfseries\color{blue}}{\thesubsection.}{4pt}{}
%\titleformat{\subsubsection}[runin]{\bfseries\color{blue}}{\thesubsubsection.}{1em}{}

\usepackage{geometry}		% change paper size
\geometry{
  a4paper,         % or letterpaper
  textwidth=18cm,  % llncs has 12.2cm
  textheight=27cm, % llncs has 19.3cm
  heightrounded,   % integer number of lines
  hratio=1:1,      % horizontally centered
  vratio=2:3,      % not vertically centered
}
\usepackage[fontsize=13pt]{scrextend}

\newcommand*\NewSym[1]{\vcenter{\hbox{\includegraphics{#1}}}}
\newcommand{\dashh}{\textemdash~}
\newcommand{\english}[1]{\mbox{\textit{#1}}}
\newcommand{\tab}{\hspace*{2cm}}

% ***** Boxed variables inside math equations
% \newcommand*{\boxedcolor}{black}
\makeatletter
% \renewcommand{\boxed}[1]{\textcolor{\boxedcolor}{%
% \fbox{\normalcolor\m@th$\displaystyle#1$}}}
% \setlength{\fboxsep}{1pt}
\renewcommand{\boxed}[1]{\fbox{\m@th$\displaystyle\scalebox{0.9}{#1}$} \,}
\makeatother

\overfullrule=0mm

\newsavebox{\MyName}
\savebox{\MyName}{\includegraphics[scale=0.6]{YKY.png}}

\title{\cc{投诉 HK Deep Learning 群组踢人}{HK Deep Learning group's unreasonable banning of membership}}
%\normalsize{-- a minimalist cognitive architecture combining\\
%reinforcement learning and deep learning}}
\titlerunning{投诉 HK Deep Learning 群组}
\author{\usebox{\MyName} (King-Yin Yan)
% \\ \footnotesize{General.Intelligence@Gmail.com}
%\and
%Ben Goertzel
%\and
%Juan Carlos Kuri Pinto
}
\institute{General.Intelligence@Gmail.com}
\date{\today}

\begin{document}
\let\labelitemi\labelitemii

\maketitle

\noindent
\makebox[\linewidth]{\small \today}

%\setlength{\parindent}{0em}
\setlength{\parskip}{2.8ex plus0.8ex minus0.8ex}
% \setlength{\parskip}{2.8ex}

\begin{abstract}
\cc{
	在
}{
	An intelligent agent needs the ability to access its own knowledge, which comes for free in classical logic-based AI, but neural networks are notorious for the ``black-box'' problem.  The solution is to have the network act on its own weights.
}
\end{abstract}

%\begin{keywords}
%reinforcement learning, control theory, deep learning, cognitive architecture
%\end{keywords}

\setcounter{section}{-1}
\section{Introduction}

这件事件表面上似乎不是种族歧视,但我想打个比喻: 有个画家上下倒转来画人像,当他完成之后,将画布倒转过来,所有观众立刻辨认出那是人像,而且是那人像是 李小龙,unmistakably。 而我现在提出的这个 case,我觉得如果用了某个 non-obvious 的角度来看,其实可以看到这个 case 明显是种族歧视,而并不是牵强的,只不过这个视角是 non-obvious。 

在这件事件中我(基本上)没有做错,但我被踢了出群组,蒙受损失,而这个决定是 Joseph 做的。  他踢我出群组的原因是因为我学习了外国的技术,而且比他早,他稍为落后,而这一点令他不满。  然而,外国人也是人,他们比香港/中国进步,这一点所有香港人,包括 Joseph,都必需接受,whether willing or unwilling。  而 Joseph 在他的群组经常分享和赞同 外国最新的研究成果。  甚至他自己也说他喜欢积极地学习。  他踢我出群组是因为他不愿意承认我努力的成果,因为他似乎觉得所有香港人都应该「在同一起跑线上」。  但问题是他这个要求是不合理的,我只是做了我一直以来比较有专长的方向; 在香港进步的过程中,必然有些人较先进,有些人较落后,但他不容纳我比其他香港人较进步。  另一个比喻: 同样是女人,但当一个女人做了高级行政人员之后她就被革职,这性别歧视是成立的。 


\section{关於投诉人 甄景贤}

\renewcommand\labelenumi{(\theenumi)}
\begin{enumerate}
	\item \textbf{abc}
\end{enumerate}

\section{关於被告 Joseph Cheng}

%\titleformat*{\subsection}[runin]{}{\thesubsection}{}{}[]
%\titleformat*{\subsection}[runin]{\normalsize\textbf}{\thesubsection}

\subsection{} 我在 2011年 11月 认识 Joseph,当时是在 创业活动 "Startup Weekend" 中的同一团队,当时我问 Joseph 他的技术背景,发现他原来没有上大学,他职业是自己开公司帮人写网页,而他当时连 recursion 是什么都不知道(那是电算机科学中的基本原理,本科生都应该懂的),也不懂得微积分;  我当时的反应是有点失望,觉得香港的技术人才水平不够高。  然而,到了 2017 年,我亲眼目睹 Joseph 已经变成了香港 deep learning(即「深度学习」,人工智能中现时最火的技术)方面的一个专家,对於他的进步我觉得很欣喜,同时我察觉到 Joseph 这个人很聪明。

\subsection{} From 2012 May - 2014, 我曾经雇用 Joseph 写一些 AI 程式,同时我有教 Joseph 一些 AI 方面的知识。 这时 Joseph 开始对 AI 感到很有兴趣。  我一直把 Joseph 当作 研发的合作夥伴。

\subsection{} 我在 2012 年开始,组织了HK "Machine Learners" 群组,交流人工智能方面的学术,并且租用某些场地 做研讨会,大约做了 1-2 年时间,但反应不热烈(可能是数学方面比较高深)。 后来 Joseph 在 2017 年开设类似群组,我为了不和他「抢人流」,所以没有提出反对,而是加入了他们的群组。 

\section{对精神病的歧视}

\subsection{我曾因精神病住院} 在 2009 年底,我因为和家人吵架,在不太情愿的情况下进了(政府的)精神病院,经医生们诊断后,认为我有精神分裂症,被强制住院 3 星期,之后出院,在家疗养及吃药、定期覆诊,大约一年后停药。    

\subsection{} 覆诊时,精神科医生向我表示我有权申请香港政府给残疾人士的福利,但我觉得自己仍有能力赚钱和照顾自己,所以没有申请,但在政府的医疗记录上有关於我的精神病纪录。 

\subsection{我的精神病成因} 其实我的精神病问题是 paranoia(被害妄想症),亦即我经常怀疑别人在监视我,令我情绪上很困扰。   但我自己亦意识到这个 paranoia 可能并不实际存在,所以只是一种 \textbf{怀疑}。 例如某些男人会病态地怀疑自己的妻子不贞,但如果那人的妻子是比较风骚的话,则这种怀疑亦未必算是病态的,甚至那妻子的不忠可能是真的。  这一点很重要,因为严格来说,我平日的表现没有不合常理的行为,只是内心很不快乐,但这个「病」并没有太大地影响我的日常工作。  精神科医生亦对我说,我在同类的病人中是有较高的 cognitive function。 覆诊期间,我在某国际人工智能 journal 上发表了一篇论文,覆诊医生觉得我在生活上大有进步。 

\subsection{我写了一本关於自己精神病经历的书} 自 2013 Sept 开始,我在网上发表《妄想症》一书,内容大致上有几点:

\begin{subsub}
描述我在美国留学后开始有 paranoia 的经过
\end{subsub}

\begin{subsub}
我对我的美国堂妹迷恋但遭拒绝的经过; in particular,我在 1990's 年代末,结识美国堂妹,对她在文化上的优越感,我觉得很羡慕又妒忌。  后来被拒绝之后我感到恼羞成怒,几乎想杀了她报复,这是我一生中感到最极端的愤怒情绪。 我之所以在此提到这点,是因为我发觉很多香港人接触到美国文化中的进步思想,都会产生不适应的情况,而对那些已经 Westernize 的人产生极度怨恨,包括 Joseph 都是属於这一情况,所以这点是有关的。
\end{subsub}

\section{对香港人的歧视}

%\bibliographystyle{unsrtnat} % or number or aaai ...
%\bibliography{AGI-book}

\end{document}
